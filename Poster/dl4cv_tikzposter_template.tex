\documentclass[25pt, a0paper, landscape]{tikzposter}
\tikzposterlatexaffectionproofoff
\usepackage[utf8]{inputenc}
\usepackage{authblk}
\makeatletter
\renewcommand\maketitle{\AB@maketitle} % revert \maketitle to its old definition
\renewcommand\AB@affilsepx{\quad\protect\Affilfont} % put affiliations into one line
\makeatother
\renewcommand\Affilfont{\Large} % set font for affiliations
\usepackage{amsmath, amsfonts, amssymb}
\usepackage{tikz}
\usepackage{pgfplots}
% align columns of tikzposter; needs two compilations
\usepackage[colalign]{column_aligned}

% tikzposter meta settings
\usetheme{Default}
\usetitlestyle{Default}
\useblockstyle{Default}

%%%%%%%%%%% redefine title matter to include one logo on each side of the title; adjust with \LogoSep
\makeatletter
\newcommand\insertlogoi[2][]{\def\@insertlogoi{\includegraphics[#1]{#2}}}
\newcommand\insertlogoii[2][]{\def\@insertlogoii{\includegraphics[#1]{#2}}}
\newlength\LogoSep
\setlength\LogoSep{-70pt}

\renewcommand\maketitle[1][]{  % #1 keys
    \normalsize
    \setkeys{title}{#1}
    % Title dummy to get title height
    \node[inner sep=\TP@titleinnersep, line width=\TP@titlelinewidth, anchor=north, minimum width=\TP@visibletextwidth-2\TP@titleinnersep]
    (TP@title) at ($(0, 0.5\textheight-\TP@titletotopverticalspace)$) {\parbox{\TP@titlewidth-2\TP@titleinnersep}{\TP@maketitle}};
    \draw let \p1 = ($(TP@title.north)-(TP@title.south)$) in node {
        \setlength{\TP@titleheight}{\y1}
        \setlength{\titleheight}{\y1}
        \global\TP@titleheight=\TP@titleheight
        \global\titleheight=\titleheight
    };

    % Compute title position
    \setlength{\titleposleft}{-0.5\titlewidth}
    \setlength{\titleposright}{\titleposleft+\titlewidth}
    \setlength{\titlepostop}{0.5\textheight-\TP@titletotopverticalspace}
    \setlength{\titleposbottom}{\titlepostop-\titleheight}

    % Title style (background)
    \TP@titlestyle

    % Title node
    \node[inner sep=\TP@titleinnersep, line width=\TP@titlelinewidth, anchor=north, minimum width=\TP@visibletextwidth-2\TP@titleinnersep]
    at (0,0.5\textheight-\TP@titletotopverticalspace)
    (title)
    {\parbox{\TP@titlewidth-2\TP@titleinnersep}{\TP@maketitle}};

    \node[inner sep=0pt,anchor=west] 
    at ([xshift=-\LogoSep]title.west)
    {\@insertlogoi};

    \node[inner sep=0pt,anchor=east] 
    at ([xshift=\LogoSep]title.east)
    {\@insertlogoii};

    % Settings for blocks
    \normalsize
    \setlength{\TP@blocktop}{\titleposbottom-\TP@titletoblockverticalspace}
}
\makeatother
%%%%%%%%%%%%%%%%%%%%%%%%%%%%%%%%%%%%%


% color handling
\definecolor{TumBlue}{cmyk}{1,0.43,0,0}
\colorlet{blocktitlebgcolor}{TumBlue}
\colorlet{backgroundcolor}{white}

% title matter
\title{Object Detection within a Robotic Application}

\author[1]{Chia-Wen Tsai}
\author[2]{Felizitas Kunz}
\author[1]{Christoph Caprano}
\author[1]{Oskar Haller}

\affil[1]{Technical University of Munich}
\affil[2]{Ludwig-Maximilians-University, Munich}

\insertlogoi[width=15cm]{tum_logo}
\insertlogoii[width=15cm]{tum_logo}

% main document
\begin{document}

\maketitle

\begin{columns}
    \column{0.5}
    \block{Introduction}{Our idea was to teach a Braccio Robotic Arm to play the child's game pairs. It should detect the playing cards laying on the table, their position and the motif of a card and find the second one. The robot picks one playing card up, places it on the stack and searches for the second one within the remaining cards. For object detection YOLOv2-Real-Time-Object detection is used for transfer learning with datasets containing images of our playing cards. We trained our network with Google Cloud Service.}
    \block{Dataset}{\textit{PASCAL VOC Dataset + our images?}}
    \block{Images}{\item network from tiny yolo
    	\item 10 cards + classes
    	\item architecture
    	\item Filters from results
    	\item Camera input with detection labels}

    \column{0.5}
    \block{Related Work}{For the object detection, we used Yolo9000 with transfer learning to detect the motifs of the playing cards in real time. \textit{Add table? Add link?}\\The communication between the Braccio Robotic Arm and the computer is enabled with ROS. \textit{Also with the Arduino?}}
    \block{Data and Statistics}{Content in your block.}
    \block{Methodology}{\begin{itemize}
    		\item Input stream from camera
    		\item Send stream to computer
    		\item Pre-trained Yolo detects motifs of playing cards
    		\item If card is detected, command is sent to robot to pick it up with help of a magnet and a metal strip attached on the back of the card
    		\item Card is put on stack
    		\item Robot searches for card with the same motif
    		\item If detected, it picks it up and places it on the stack
    		\item Robot starts looking for a new pair of playing cards
    	\end{itemize}\textit{Description of network architecture?} \textit{How well did our model perform during training?}}
\end{columns}

\end{document}
